\chapter{Implementation}
\label{implementation}
This chapter details on the implementation of the concepts discussed in previous chapter for route planner and trajectory planner modules. All the modules are implemented as independent nodes in Robot Operating System (ROS) and communicate with each other using ROS messages. Figure \ref{message_flow} describes the communication across different modules. 

\begin{figure}[H]
	\centering
	\includegraphics[width=0.9\textwidth]{Images/implementation/message_flow.png}
	\caption{Message Flow across Modules}
	\label{message_flow}
\end{figure}


\section{Route Planner}
Route planner parses the RNDF file and stores way points, their relation to lanes and further to segments. The below code segment shows the declaration of the way point, here coordi indicates the coordinates of way points, idn is the unique id for each way point and parent is the lane to which way point belongs. Similarly there lane stores set of way points, parent is the segment to which the lane belongs. 

\begin{lstlisting}
class c_waypoint(object):
	def __init__(self,name,coordi,parent, idn):
		self.name = name
		self.coordi = coordi
		self.parent = parent
		self.idn = idn
	def __str__(self):
		return '{}'.format(self.name)

\end{lstlisting}

\begin{figure}
	\centering
	\includegraphics[width=0.8\textwidth]{Images/implementation/route_planner.png}
	\caption{Route Planner Functions}
	\label{route_planner_func}
\end{figure}

\section{Motion Planner}
Motion Planner is sub divided into three sub classes names motion planner, vehicle path and vehicle state. Vehicle path class stores the path to be traversed, performs all the conversions from Cartesian frame to Frenet frame and vice versa. The module also calculates speed limit based on the curvature of the road. Figure \ref{vehicle_path_class} further details on the structure of the vehicle path class. 

Vehicle state class maintains the information regarding current position and, orientation from odometry messages, and list of obstacles including information about their position, orientation, speed, size. In every planning cycle, initially a snapshot of vehicle state is used to create trajectories and evaluate them to remove affects of changes in states in one planning cycle. Figure \ref{vehicle_state_class} details further on structure of the Vehicle state class. 


Motion planner class is the core of the planner, it initially creates samples, assigns pre-costs to these samples, creates trajectories based on the samples and pre costs, evaluates the trajectories find the best trajectory which is collision free and closer to destination. Once the best trajectory is found it converts to the format specific to the controller and transmits it. Figure \ref{motion_planner_class} describes the interactions between different functions in motion planning.


\begin{figure}
	\centering
	\includegraphics[width=0.8\textwidth]{Images/implementation/vehicle_path.png}
	\caption{Vehicle Path Class}
	\label{vehicle_path_class}
\end{figure}

\begin{figure}
	\centering
	\includegraphics[width=0.8\textwidth]{Images/implementation/vehicle_state.png}
	\caption{Vehicle State Class}
	\label{vehicle_state_class}
\end{figure}

\begin{figure}
	\centering
	\includegraphics[width=0.8\textwidth]{Images/implementation/motion_planner.png}
	\caption{Motion Planner Functions}
	\label{motion_planner_class}
\end{figure}






\iffalse

Code for storing 

class c_exit(object):
def __init__(self, entry, exit, parent):
self.entry = entry
self.exit = exit
self.parent = parent

class c_lane(object):
def __init__(self,name,parent,idn):
self.name = name
self.parent = parent
self.idn = idn
self.waypoints = []
self.exits = []
def add_waypoint(self,waypoint):
self.waypoints.append(waypoint)
def add_exit(self,exit):
self.exits.append(exit)
def __str__(self):
return '{}'.format(self.name)

class c_segment(object):
def __init__(self,name,parent,idn):
self.name = name
self.lanes = []
self.parent = parent
self.idn = idn
def add_lane(self,lane):
self.lanes.append(lane)
def __str__(self):
return '{}'.format(self.name)

class c_rndf(object):
def __init__(self,name):
self.name = name
self.segments= []
def add_segment(self,segment):
self.segments.append(segment)
def __str__(self):
return '{}'.format(self.name)
Check if this chapter is needed, 

Write a flow chart describing various functions implemented 

How the collision check is implemented

Further details on how these trajectories are converted to x,y coordinates used by the trajectory follower and the constraints in implementation for model car will be discussed in

How costs are added

Which messages are used for what, who sends what and who receives what - flow diagrams

speak about splines, how the state machine works, write a flowchart etc

Inform when which order splines are good to implemented

need for adding extra costs if this has to be ported for full scale cars

Issues - 
\fi