\chapter*{Abstract}
The goal of any motion planner is to achieve a driving trajectory that is collision free, smooth and responsive (reactive), at the same time being far-sighted to maintain consistency, deliberative and allow smooth transitions. Motion planning for robots is a widely researched topic in conventional robotics in unstructured environments, and dynamic environments are pursued pseudo static. Similarly, trajectory planning for autonomous vehicles is an extensively studied topic with computationally expensive planning techniques. This thesis proposes a new planning technique suitable for model cars running in structured environments and low computational devices. This thesis offers a hierarchical sampling based algorithm designed especially for small robotic vehicles operating in structured dynamic urban environments. The first layer in motion planner creates a reference path from an obstacle-free road network definition file. The next layer creates multiple forward projecting samples (acceleration and lateral motion), and a pre-cost is assigned to them based on their closeness to the expected behaviour of ego vehicle. Trajectories are generated and evaluated in the lowest first order of the samples till the top of the stack has the lowest cost evaluated trajectory. This reduces the need to evaluate hundreds of trajectories when the trajectory with expected behaviour is found. 

The next step in motion planning is to evaluate that the created trajectories are collision-free with respect to static and dynamic obstacles in the scene. This thesis employs a two-step collision detection for static obstacles by checking maximum of four points on the trajectory to decide whether the trajectory is collision free. This is possible by exploiting the properties of the polynomials used in trajectory representation and path modelling. For collision detection with dynamic obstacles, the approach to test for static obstacles is extended to check for collision in time when there is a collision in space for the ego vehicle and the obstacle. It is sufficient to check for the time gap between the ego and the obstacle only at the borders of the collision area to determine whether the trajectory is collision free. 

The proposed planner has been verified by testing it in various scenarios that are encountered in urban driving. The planner was able to generate feasible and reactive trajectories that are collision free in different scenarios and able to run on low computational platforms such as modelcar. 
