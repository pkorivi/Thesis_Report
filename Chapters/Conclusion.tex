\chapter{Conclusions and Future Work}
\label{conclusion}
\section{Conclusions}

\section{Future Work}
Replace by smoother polynomials over splines, especially in curves and when not following centre lane, they tend to be very bad. 

// Diss shui thesis - read though page 80 and understand further on benefits and demerits of polynomials vs splines. Add some in evaluation and some in future work


Prediction of state from where the planner should start planning instead of current position. 
Due to inaccuracies in current planners measurement of speed and acceleration it is tough to estimate where the vehicle will be when the planner is under execution. Currently based on assumption that the vehicle will follow the current path for next few ms, it is made offset in control node. This can be improved to have better synchronizaton between planner and controller. 


Create two functions to map lateral shift as a function of time and distance based on speed over current function only mapping based on distance.  

Improve the lateral shift function to have an option to include where the ego vehicle is in current state with respect to lane change. If lets say the ego vehicle creates a trajectory to go from ds to dt in 5s, then when the trajectory is evaluated next time the ego vehicle is at d1 and it will choose 5s to travel to dt from there thus delaying the lane shift and this process repeats indefinitely and the robot will never exactly reach the target d. Thus the d\_final for lane shift from previous trajectory should be associated with the s\_final previous thus making the trajectory to be executed in the required distance ahead. This is similar to classical Proportional controller issue.  

Use quintic polynomials at high speeds and cubic at low speeds. 

Improve collision checking with respect to pedestrians following the huge research published in this specific domain of pedestrian tracking in autonomous cars. 