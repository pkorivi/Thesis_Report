\chapter*{Abstrakt}
Das Ziel eines jeden Bewegungsplaners ist es, eine kollisionsfreie, glatte und reaktionsfähige (reaktive) Fahrtrajektorie zu erreichen, die zugleich weitsichtig ist, um Konsistenz zu bewahren, zu beraten und sanfte Übergänge zu ermöglichen. Die Bewegungsplanung für Roboter ist ein weit erforschtes Thema in der konventionellen Robotik in unstrukturierten Umgebungen. In ähnlicher Weise ist auch die Trajektorienplanung für autonome Fahrzeuge ein umfassend untersuchtes Thema mit rechenintensiven Planungstechniken. Diese Arbeit schlägt eine neue Planungstechnik vor, die für Modellautos geeignet ist, die in strukturierten Umgebungen und mit wenig Rechengeräten arbeiten, um ein Verhalten zu erzeugen, das dem von völlig autonomen Fahrzeugen ähnelt. Diese Arbeit bietet einen hierarchischen, stichprobenbasierten Algorithmus, der speziell für kleine Roboterfahrzeuge entwickelt wurde, die in strukturierten, dynamischen städtischen Umgebungen arbeiten. Die erste Ebene im Bewegungsplaner erstellt einen Referenzpfad aus einer Straßendokumentationsdatei ohne Hindernisse. Die nächste Schicht erzeugt mehrere vorwärtsprojizierende Proben (Beschleunigung und laterale Bewegung), und ihnen werden Vorkosten zugeordnet, basierend auf ihrer Nähe zum erwarteten Verhalten des Ego-Fahrzeugs. Flugbahnen werden in der niedrigsten ersten Ordnung der Proben erzeugt und ausgewertet, bis die Oberseite des Stapels die niedrigste Kosten-bewertete Flugbahn aufweist. Dies verringert die Notwendigkeit, Hunderte von Trajektorien zu bewerten, wenn die Trajektorie mit dem erwarteten Verhalten gefunden wird.

Der nächste Schritt in der Bewegungsplanung besteht darin, zu bewerten, dass die erzeugten Trajektorien kollisionsfrei in Bezug auf statische und dynamische Hindernisse in der Szene sind. Diese Arbeit verwendet eine zweistufige Kollisionserkennung für statische Hindernisse durch die Überprüfung von maximal vier Punkten auf der Trajektorie, um zu entscheiden, ob die Trajektorie kollisionsfrei ist. Dies ist möglich, indem die Eigenschaften der Polynome, die bei der Trajektorienrepräsentation und der Pfadmodellierung verwendet werden, ausgenutzt werden. Zur Kollisionserkennung mit dynamischen Hindernissen wird der Ansatz zur Prüfung auf statische Hindernisse erweitert, um rechtzeitig auf eine Kollision zu prüfen, wenn es zu einer Kollision im Raum für das Ego-Fahrzeug und das Hindernis kommt. Es ist ausreichend, den Zeitabstand zwischen dem Ich und dem Hindernis nur an den Grenzen des Kollisionsbereichs zu prüfen, um festzustellen, ob die Trajektorie kollisionsfrei ist.

Der vorgeschlagene Planer wurde verifiziert, indem er in verschiedenen Szenarien getestet wurde, die im Stadtverkehr auftreten. Der Planer war in der Lage, machbare und reaktive Trajektorien zu erzeugen, die in verschiedenen Szenarien in dynamischen Umgebungen kollisionsfrei sind und auf niedrigen rechnergestützten Plattformen wie Modellauto laufen können.


